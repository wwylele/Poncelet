% In this file you should put the actual content of the blueprint.
% It will be used both by the web and the print version.
% It should *not* include the \begin{document}
%
% If you want to split the blueprint content into several files then
% the current file can be a simple sequence of \input. Otherwise It
% can start with a \section or \chapter for instance.
\chapter{Introduction}

This is a formalization of an "elementary" proof of \href{https://en.wikipedia.org/wiki/Poncelet%27s_closure_theorem}{Poncelet's closure theorem}:
it only involves basic algebra and group laws of elliptic curves. It is more "low level" comparing
to most modern proof of the theorem, where explicit curves and maps are constructed.

\section{The theorem statement}

Given two plane conics {\it Outer} and {\it Inner}, if it is possible to find a $n$-sided polygon
simultaneously inscribed in {\it Outer} and circumscribed around {\it Inner}, then it is possible
to find another such polygon, one of whose vertices is an arbitrary point on {\it Outer}.

\section{Notes}

The theorem is usually considered in the real or complex projective plane, but as the arguments are
purely algebraic, we will be able to formalize this in any underlying field with characteristic 0.

Since we are in projective plane, it is possible that the polygon will contain points at infinity
or lines at infinity.

Several degenerate cases need to be considered: degenerate conics, and degenerate intersection of 
them (tangent). Some of these degenerate cases can be naturally included in the general case, while
others needs to be discussed separately.

Even in the non-degenerate case, it is possible for a polygon to contain repeated vertices and
edges. This happens when a vertex lands on a intersection of the two conics,
or an edge is a shared tangent of the two conics. Algebraically, as we will define directed edges, 
we can distinguish between repeated edges, so this is generally not an issue. However, it can be
confusing when presented geometrically.

\chapter{Two circles}

We will first focus on a special case of the theorem: two circles. This later will turn out to be
a good representation of most general cases. To make it even simpler, we will fix {\it Inner} to be
the unit circle, and restrict the center of {\it Outer} on the x-axis
\[
Inner: x^2 + y^2 = z^2
\]
\[
Outer: (x-uz)^2 + y^2 = r^2z^2
\]
where $u$ and $r$ are the only two free parameters we are allowed to change. As mentioned before,
all variables are in a field $K$ with characteristic 0. We will also assume the existence of a
degree-2 constant $k$ such that
\[
k^2 = (u + r)^2 - 1
\]
which will be used in the next chapter. We can always consider the algebraic closure of $K$ 
to ensure the existence of $k$. However, even if we restrict ourselves in $\mathbb{R}$, we still can 
find such $k$ for all useful cases: depending on the sign we choose for $r$, $u + r$ represents 
either the left-most or the right-most point of {\it Outer}, so to say $k$ exists, we need to ensure
at least one of the two points is not inside {\it Inner}. Indeed, if both points are inside 
{\it Inner}, then {\it Outer} is completely enclosed by {\it Inner} in the real plane, making the 
polygon impossible to form.

Finally, we don't allow $u = 0$ (concentric circles) or $r = 0$ (degenerate {\it Outer}). We will
discuss these cases in a later chapter

We encode these in the following definition

\begin{definition}\label{Config}\lean{Config}\leanok
    A \emph{two-circle configuration} in field $K$ with characteristic 0 is a tuple 
    $(u, r, k)$ such that $u \ne 0$, $r \ne 0$, and $k^2 = (u + r)^2 - 1$
\end{definition}

We call a configuration \emph{singular} if the two circles are tangent to each other. That is,
$u + r = \pm 1$ or $u - r = \pm 1$. We further categorize $u + r = \pm 1$ as 
\emph{positive-singular}, and $u - r = \pm 1$ as \emph{negative-singular}. This distinction is 
inconsequential geometrically, as we can change the sign of $r$ while preserving the geometry, but
this will provide algebraic significance later.

Moving on to vertices and edges, we consider the pair 
\[
([x : y : z], [a : b : c]): \mathbf{P}(2, K) \times \mathbf{P}(2, K)
\]
where $[x : y : z]$ is the projective coordinates of a point, and $[a : b : c]$ is the projective
coordinates of a line $ax+by=cz$. We define a subset $Dom$ of 
$\mathbf{P}(2, K) \times \mathbf{P}(2, K)$ that represents legal vertex-edge pairs for the
configuration

\begin{definition}\label{dom}\lean{dom}\leanok
    The subset \emph{domain} ($\mathrm{Dom}$) of $\mathbf{P}(2, K) \times \mathbf{P}(2, K)$ consists of
    pairs $([x : y : z], [a : b : c])$ such that
    \begin{itemize}
        \item $(x-uz)^2 + y^2 = r^2z^2$
        \item $a^2 + b^2 = c^2$
        \item $ax+by=cz$
    \end{itemize}
\end{definition}

This means that the vertex is on {\it Outer}, the edge is tangent to {\it Inner}, and the vertex is
on the line of the edge. We also effectively defined directed edges: while each edge has two 
associated vertices, we specify one of them as the "starting" vertices. Dually, this also assign
a direction to vertices: we specify the "forward" edge among the two edges associated to the vertex.

We can think of the process of forming a polygon as iteratively generating new elements in $\mathrm{Dom}$
from previous ones, where the new starting vertex is the ending vertex of the previous edge, and 
the new forward edge is the edge associated with the new starting vertex, different from the previous
edge. We use the function $\mathrm{next}: \mathrm{Dom} \to \mathrm{Dom}$ to represent a such iteration.

We first recognize that $\mathrm{next}$ is a composition of two functions 
$\mathrm{rPoint}$ (reflect point) and $\mathrm{rChord}$ (reflect chord).

\begin{definition}\label{rPoint}\lean{rPoint, mapsTo_rPoint}\leanok
    $\mathrm{rPoint}: \mathrm{Dom} \to \mathrm{Dom}$ keeps the edge component, and sends the vertex to 
    the other one associated with the edge:
    \[
    \mathrm{rPoint}([x : y : z], [a : b : c]) = (V, [a : b : c])
    \]
    where
    \[
    V = \begin{cases}
        \left[-(r^2 - u^2)b : (r^2 - u^2)a : 2ub\right] & (c = 0, z = 0)\\
        \left[b : -a : 0\right] & (c = 0, z \ne 0)\\
        \left[2 acz + 2ub^2z - c^2x : 2bcz + 2uabz - c^2y : c^2z \right] & (c \ne 0)
    \end{cases}
    \]
\end{definition}

\begin{definition}\label{rChord}\lean{rChord, mapsTo_rChord}\leanok
    $\mathrm{rChord}: \mathrm{Dom} \to \mathrm{Dom}$ keeps the vertex component, and sends the edge to 
    the other one associated with the vertex:
    \[
    \mathrm{rChord}([x : y : z], [a : b : c]) = ([x : y : z], E)
    \]
    where
    \[
    E = \begin{cases}
        \left[a : -b : c\right] & (2ux +r^2z-u^2z = 0, x = 0)\\
        \left[y(z^2-y^2) : x(z^2 + y^2) : 2xyz\right] & (2ux +r^2z-u^2z = 0, x\ne0, c=0) \\
        \left[y : -x : 0\right] & (2ux +r^2z-u^2z = 0, x\ne0, c\ne0) \\
        \left[2cx - a(2ux+r^2z-u^2z) : 2cy - b(2ux+r^2z-u^2z) : c(2ux+r^2z-u^2z) \right] &
           (2ux+r^2z-u^2z \ne 0)
    \end{cases}
    \]
\end{definition}

We verify that these are legitimate by proving their characterizing lemma
\begin{lemma}
    \lean{rPoint_rPoint}
    \label{rPoint_rPoint}
    \uses{rPoint}
    \leanok
    $\mathrm{rPoint}$ is an involution.
\end{lemma}
\begin{proof}
    \leanok
    Expand the function composition and verify the equality for all cases
\end{proof}

\begin{lemma}
    \lean{rPoint_bijOn}
    \label{rPoint_bijOn}
    \uses{rPoint}
    \leanok
    $\mathrm{rPoint}$ is a bijection from $\mathrm{Dom}$ to itself.
\end{lemma}
\begin{proof}
    \leanok
    Trivial from being an involution after verifying the domain and the range
\end{proof}

\begin{lemma}
    \lean{rPoint_eq_self}
    \label{rPoint_eq_self}
    \uses{rPoint}
    \leanok
    $\mathrm{rPoint}$ maps a vertex to itself if and only if the input edge is a shared tangent to 
    {\it Inner} and {\it Outer}.
\end{lemma}
\begin{proof}
    \leanok
    The formula of $\mathrm{rPoint}$ is constructed by calculating a root of a quadratic
    equation when given another root. Two roots coincide if and only if the discriminant vanishes.
    The rest is verifying the equations.
\end{proof}

\begin{lemma}
    \lean{rChord}
    \label{rChord_rChord}
    \uses{rChord}
    \leanok
    $\mathrm{rChord}$ is an involution.
\end{lemma}
\begin{proof}
    \leanok
    Expand the function composition and verify the equality for all cases
\end{proof}

\begin{lemma}
    \lean{rChord_bijOn}
    \label{rChord_bijOn}
    \uses{rChord}
    \leanok
    $\mathrm{rChord}$ is a bijection from $\mathrm{Dom}$ to itself.
\end{lemma}
\begin{proof}
    \leanok
    Trivial from being an involution after verifying the domain and the range
\end{proof}

\begin{lemma}
    \lean{rChord_eq_self}
    \label{rChord_eq_self}
    \uses{rChord}
    \leanok
    $\mathrm{rChord}$ maps an edge to itself if and only if the input vertex is on {\it Inner}.
    In other words, it is an intersection of {\it Inner} and {\it Outer}.
\end{lemma}
\begin{proof}
    \leanok
    The formula of $\mathrm{rChord}$ is constructed by calculating a root of a quadratic
    equation when given another root. Two roots coincide if and only if the discriminant vanishes.
    The rest is verifying the equations.
\end{proof}

Using these, we define the function $\mathrm{next}$

\begin{definition}\label{next}\lean{next}\leanok
    $\mathrm{next}: \mathrm{Dom} \to \mathrm{Dom}$ sends a pair of vertex and edge to the next one on the polygon.
    \[
        \mathrm{next}(V, E) = \mathrm{rChord}(\mathrm{rPoint}(V, E))
    \]
\end{definition}
It is obvious that
\begin{lemma}
    \lean{next_bijOn}
    \label{next_bijOn}
    \uses{next}
    \leanok
    $\mathrm{next}$ is a bijection from $\mathrm{Dom}$ to itself.
\end{lemma}
\begin{proof}
    \leanok
    $\mathrm{next}$ is a composition of two bijections
\end{proof}

We call $(V, E)$ \emph{singular} if $\mathrm{next}(V, E) = (V, E)$. This is important because applying
$\mathrm{next}$ on a singular point repeatedly will result in a degenerate polygon where all vertices coincide.
Obviously this doesn't tell us how other vertices behaves, so we will need to exclude it from the theorem.
We characterize singular points with the following lemma
\begin{lemma}
    \lean{next_eq_self, next_eq_self'}
    \label{next_eq_self}
    \uses{next}
    \leanok
    $\mathrm{next}(V, E) = (V, E)$ if and only if the $V$ is an intersection and $E$ is a shared
    tangent to {\it Inner} and {\it Outer}. Consequently, {\it Inner} and {\it Outer} must be tangent
    to each other at $(V, E) = ([1 : 0 : 1], [1 : 0 : 1])$ or at 
    $(V, E) = ([-1 : 0 : 1], [-1 : 0 : 1])$.
\end{lemma}
\begin{proof}
    The first part is directly from the property of \ref{rPoint_eq_self} and \ref{rChord_self}. 
    One can then solve the system of equations for $V$ and $E$ to get the coordinates. 
\end{proof}
This tells us that the singular point only exists for a singular configuration, is unique if exists, 
and is precisely the tangent point.

We now formalize the statement of a version of Poncelet's closure theorem
\begin{theorem}
    Given a two-circle configuration in field $K$, if for a natural number $n$, and some non-singular
    $(V, E) \in \mathrm{Dom}$ it holds that $(\mathrm{next})^n(V, E) = (V, E)$, then the same holds for 
    the same $n$ and all $(V, E) \in \mathrm{Dom}$.
\end{theorem}

\chapter{Elliptic curve}

To prove the Poncelet's closure theorem for the two-circle configuration, we will construct a bijection
between $\mathrm{Dom}$ and a certain elliptic curve $E$, and show that the function $\mathrm{next}$ maps
to an addition in the group of the elliptic curve. We define the curve $E$ as follows

\begin{definition}\label{elliptic}\lean{elliptic}\leanok
    $E$ is an affine cubic curve defined by the following equation, plus a point at infinity as the
    group identity
    \[
    Y^2 = X\left(r^2X^2 + \left(1-u^2-r^2\right)X + u^2\right)
    \]
    $E_0$ is the set of non-singular points of $E$ including the point at infinity, which 
    forms an abelian group.
\end{definition}

The curve $E$ is not always an elliptic curve because it can be singular. This is characterized by

\begin{lemma}
    \lean{singular_elliptic}
    \label{singular_elliptic}
    \uses{elliptic}
    \leanok
    The curve $E$ contains a singular point only in the following 4 cases
    \begin{itemize}
        \item $u + r = \pm 1$ (positive singular), and the singular point is $(-u / r, 0)$
        \item $u - r = \pm 1$ (negative singular), and the singular point is $(u / r, 0)$
    \end{itemize}
\end{lemma}
\begin{proof}
    \leanok
    This can be solved from the system of equations of $E$ and the singular point.
\end{proof}

\begin{corollary}
    The curve $E$ is singular if and only if the two circles are tangent to each other
\end{corollary}

In general, the group law still holds for non-singular points on a singular cubic curve, so we don't
need to consider the singular case separately for most parts.

We define three special points $o$, $w$, and $g$ on $E_0$, which will be useful soon.

\begin{definition}\label{o}\lean{o}\leanok
    \[
    o = (0, 0)
    \]
\end{definition}

\begin{definition}\label{w}\lean{w}\leanok
    \[
    w = \left(\frac{u^2}{r^2}, \frac{u^2}{r^3}\right)
    \]
\end{definition}

\begin{definition}\label{g}\lean{g}\leanok
    \[
    g = \left(1, \frac{1}{r}\right)
    \]
\end{definition}

It is easy to show that they have the following relation 
\begin{lemma}
    \lean{o_sub_w}
    \label{o_sub_w}
    \uses{o, w, g}
    \leanok
    \[
    g = o - w
    \]
    where the subtraction follows the group law of $E$
\end{lemma}
\begin{proof}
    \leanok
    Use the formula for group addition on elliptic curves
\end{proof}

Next, we define the mapping $f : E_0 \to \mathrm{Dom}$ as follows
\begin{definition}\label{f}\lean{f, mapsTo_f}\leanok
    \[
    f(X, Y) = (\mathrm{fPoint}(X, Y), \mathrm{fChord}(X, Y))
    \]
    \[
    \mathrm{fPoint}(X, Y) = \begin{cases}
        \left[u + r : 0 : 1\right] & ((X, Y) = \infty)\\
        \left[\begin{matrix}r^2 (u + r) X^2 + 2 r(1 - r^2 - ur) X + u^2(u + r):\\ 
            -2r^2kY:\\ (rX + u)^2\end{matrix}\right] & (\mbox{otherwise})
    \end{cases}
    \]
    \[
    \mathrm{fChord}(X, Y) = \begin{cases}
        \left[1 : -k : u + r\right] & ((X, Y) = \infty)\\
        \left[\begin{matrix}2 uk ((u^2-r^2)^2 + 4u^2):\\
      (r(u+r)^2 X - u((u+r)^2 - 2))((u^2 - r^2)^2 - 4 u^2):\\
      8u^2 k (u^2 - r^2) \end{matrix}\right] & (\mbox{if the next line yields $[0 : 0 : 0]$})\\
      \left[\begin{matrix}
        -2r^2((u + r)^2 - 1)(rX - u)Y +
    (rX + u)(r^2(u + r)X^2 +
      2r(1 - r(u + r))X + (u + r)u^2): \\
    -k(2r^2(rX + u)Y +
      (rX - u)(r^2(u + r)X^2 +
      2r(1 - r(u + r))X + (u + r)u^2)): \\
    (rX + u)((rX - u)^2(u + r)^2 + 4urX)
    \end{matrix}\right]
    \end{cases}
    \]
\end{definition}
As a reminder, $k^2 = (u+r)^2 - 1$. It doesn't matter whether we pick $k$ or $-k$, as changing
the sign of $k$ results in mirroring around the x-axis.

We should also note that when $k = 0$, which corresponds to the positive singular case, this
mapping becomes trivial.
\begin{lemma}
    If $k = 0$, $f$ becomes a constant function
    \[
    f(X, Y) = ([u + r : 0 : 1], [u + r : 0 : 1])
    \]
\end{lemma}
In such case, $f$ is not useful, but we can flip the sign of $r$, which keeps both $\mathrm{Dom}$
and $E$ the same, and turn it into a negative singular case and get a useful $f$. This has no
impact on the next few lemma, but this will appear again when we need the surjectivity of $f$. 

With the map $f$, we can make connection between the elliptic curve $E$ and the two-circle
configuration with the following theorems
\begin{theorem}
    \lean{f_o_sub}
    \label{f_o_sub}
    \uses{f, o, rChord}
    For all $p \in E_0$,
    \[
    f(o - p) = \mathrm{rChord}(f (p))
    \]
    \leanok
\end{theorem}
\begin{proof}
    \leanok
    Expand the formula for $f$, $\mathrm{rChord}$, and subtraction for elliptic curve.
\end{proof}

\begin{theorem}
    \lean{f_w_sub}
    \label{f_w_sub}
    \uses{f, w, rPoint}
    For all $p \in E_0$,
    \[
    f(w - p) = \mathrm{rPoint}(f (p))
    \]
    \leanok
\end{theorem}
\begin{proof}
    \leanok
    Expand the formula for $f$, $\mathrm{rPoint}$, and subtraction for elliptic curve.
\end{proof}

\begin{theorem}
    \lean{f_add_g}
    \label{f_add_g}
    \uses{f, g, next}
    For all $p \in E_0$,
    \[
    f(p + g) = \mathrm{next}(f (p))
    \]
    \leanok
\end{theorem}
\begin{proof}
    \leanok
    Use $g = o - w$ and $\mathrm{next} = \mathrm{rChord} \circ \mathrm{rPoint}$.
\end{proof}


\chapter{Two conics}