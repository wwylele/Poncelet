% In this file you should put the actual content of the blueprint.
% It will be used both by the web and the print version.
% It should *not* include the \begin{document}
%
% If you want to split the blueprint content into several files then
% the current file can be a simple sequence of \input. Otherwise It
% can start with a \section or \chapter for instance.
\chapter{Introduction}

This is a formalization of an "elementary" proof of \href{https://en.wikipedia.org/wiki/Poncelet%27s_closure_theorem}{Poncelet's closure theorem}:
it only involves basic algebra and group laws of elliptic curves. It is more "low level" comparing
to most modern proof of the theorem, where explicit curves and maps are constructed.

\section{The theorem statement}

Given two plane conics {\it Outer} and {\it Inner}, if it is possible to find a $n$-sided polygon
simultaneously inscribed in {\it Outer} and circumscribed around {\it Inner}, then it is possible
to find another another such polygon, one of whose vertices is an arbitrary point on {\it Outer}.

\section{Notes}

The theorem is usually considered in the real or complex projective plane, but as the arguments is
purely algebraic, we will be able to formalize this in any underlying field with characteristic 0.

Since we are in projective plane, it is possible that the polygon will contain points at infinity
or lines at infinity.

Several degenerate cases need to be considered: degenerate conics, and degenerate intersection of 
them (tangent). Some of these degenerate cases can be naturally included in the general case, while
others needs to be discussed separately.

Even in the non-degenerate case, it is possible for a polygon to contain repeated vertices and
edges. This happens when a vertex lands on a intersection of the two conics,
or an edge is a shared tangent of the two conics. Algebraically, as we will define directed edges, 
we can distinguish between repeated edges, so this is generally not an issue. However, it can be
confusing when presented geometrically.

\chapter{Two circles}

We will first focus on a special case of the theorem: two circles. This later will turn out to be
a good representation of most general cases. To make it even simpler, we will fix {\it Inner} to be
the unit circle, and restrict the center of {\it Outer} on the x-axis
\[
Inner: x^2 + y^2 = z^2
\]
\[
Outer: (x-uz)^2 + y^2 = r^2z^2
\]
where $u$ and $r$ are the only two free parameters we are allowed to change. As mentioned before,
all variables are in a field $K$ with characteristic 0. We will also assume the existence of a
degree-2 constant $k$ such that
\[
k^2 = (u + r)^2 - 1
\]
which will be used in the next chapter where. We can always consider the algebraic closure of $K$ 
to ensure the existence of $k$. However, even if we restrict ourselves in $\mathbb{R}$, we still can 
find such $k$ for all useful cases: depending on the sign we choose for $r$, $u + r$ represents 
either the left-most or the right-most point of {\it Outer}, so to say $k$ exists, we need to ensure
at least one of the two points is not inside {\it Inner}. Indeed, if both points are inside 
{\it Inner}, then {\it Outer} is completely enclosed by {\it Inner} in the real plane, making the 
polygon impossible to form.

Finally, we don't allow $u = 0$ (concentric circles) or $r = 0$ (degenerate {\it Outer}). We will
discuss these cases in a later chapter

We encode these in the following definition

\begin{definition}\label{Config}\lean{Config}\leanok
    A \emph{two circles configuration} in field $K$ is a tuple $(u, r, k)$ such that $u \ne 0$,
    $r \ne 0$, and $k^2 = (u + r)^2 - 1$
\end{definition}

Moving on to vertices and edges, we consider the pair 
\[
([x : y : z], [a : b : c]): \mathbf{P}(2, K) \times \mathbf{P}(2, K)
\]
where $[x : y : z]$ is the projective coordinates of a point, and $[a : b : c]$ is the projective
coordinates of a line $ax+by=cz$. We define a subset $Dom$ of 
$\mathbf{P}(2, K) \times \mathbf{P}(2, K)$ that represents legal vertex-edge pairs for the
configuration

\begin{definition}\label{dom}\lean{dom}\leanok
    The subset \emph{domain} ($\mathrm{Dom}$) of $\mathbf{P}(2, K) \times \mathbf{P}(2, K)$ consists of
    pairs $([x : y : z], [a : b : c])$ such that
    \begin{itemize}
        \item $(x-uz)^2 + y^2 = r^2z^2$
        \item $a^2 + b^2 = c^2$
        \item $ax+by=cz$
    \end{itemize}
\end{definition}

This means that the vertex is on {\it Outer}, the edge is tangent to {\it Inner}, and the vertex is
on the line of the edge. We also effectively defined directed edges: while each edge has two 
associated vertices, we specify one of them as the "starting" vertices. Dually, this also assign
a direction to vertices: we specify the "forward" edge among the two edges associated to the vertex.

We can think of the process of forming a polygon as iteratively generating new elements in $\mathrm{Dom}$
from previous ones, where the new starting vertex is the ending vertex of the previous edge, and 
the new forward edge is the edge associated with the new starting vertex, different from the previous
edge. We use the function $\mathrm{next}: \mathrm{Dom} \to \mathrm{Dom}$ to represent a such iteration.

To define the $\mathrm{next}$ function, we first recognize that it is a composition of two functions 
$\mathrm{rPoint}$ (reflect point) and $\mathrm{rChord}$ (reflect chord).

\begin{definition}\label{rPoint}\lean{rPoint}\leanok
    $\mathrm{rPoint}: \mathrm{Dom} \to \mathrm{Dom}$ keeps the edge component, and sends the vertex to 
    the other one associated with the edge:
    \[
    \mathrm{rPoint}([x : y : z], [a : b : c]) = (V, [a : b : c])
    \]
    where
    \[
    V = \begin{cases}
        \left[-(r^2 - u^2)b : (r^2 - u^2)a : 2ub\right] & (c = 0, z = 0)\\
        \left[b : -a : 0\right] & (c = 0, z \ne 0)\\
        \left[2 acz + 2ub^2z - c^2x : 2bcz + 2uabz - c^2y : c^2z \right] & (c \ne 0)
    \end{cases}
    \]
\end{definition}

\begin{definition}\label{rChord}\lean{rChord}\leanok
    $\mathrm{rChord}: \mathrm{Dom} \to \mathrm{Dom}$ keeps the vertex component, and sends the edge to 
    the other one associated with the vertex:
    \[
    \mathrm{rChord}([x : y : z], [a : b : c]) = ([x : y : z], E)
    \]
    where
    \[
    E = \begin{cases}
        \left[a : -b : c\right] & (2ux +r^2z-u^2z = 0, x = 0)\\
        \left[y(z^2-y^2) : x(z^2 + y^2) : 2xyz\right] & (2ux +r^2z-u^2z = 0, x\ne0, c=0) \\
        \left[y : -x : 0\right] & (2ux +r^2z-u^2z = 0, x\ne0, c\ne0) \\
        \left[2cx - a(2ux+r^2z-u^2z) : 2cy - b(2ux+r^2z-u^2z) : c(2ux+r^2z-u^2z) \right] &
           (2ux+r^2z-u^2z \ne 0)
    \end{cases}
    \]
\end{definition}

We verify that these are legitimate by proving their characteristic lemma
\begin{lemma}
    \lean{rPoint_rPoint}
    \label{rPoint_rPoint}
    \uses{rPoint}
    \leanok
    $\mathrm{rPoint}$ is an involution.
\end{lemma}
\begin{proof}
    Expand the function composition and verify the equality for all cases
\end{proof}

\begin{lemma}
    \lean{rPoint_bijOn}
    \label{rPoint_bijOn}
    \uses{rPoint}
    \leanok
    $\mathrm{rPoint}$ is a bijection from $\mathrm{Dom}$ to itself.
\end{lemma}
\begin{proof}
    Trivial from beingan involution after verifying the domain and the range
\end{proof}

\begin{lemma}
    \uses{rPoint}
    $\mathrm{rPoint}$ maps a vertex to itself if and only if the input edge is a shared tangent to 
    {\it Inner} and {\it Outer}.
\end{lemma}
\begin{proof}
    TODO
\end{proof}

\begin{lemma}
    \lean{rChord}
    \label{rChord_rChord}
    \uses{rChord}
    \leanok
    $\mathrm{rChord}$ is an involution.
\end{lemma}
\begin{proof}
    Expand the function composition and verify the equality for all cases
\end{proof}

\begin{lemma}
    \lean{rChord_bijOn}
    \label{rChord_bijOn}
    \uses{rChord}
    \leanok
    $\mathrm{rChord}$ is a bijection from $\mathrm{Dom}$ to itself.
\end{lemma}
\begin{proof}
    Trivial from beingan involution after verifying the domain and the range
\end{proof}

\begin{lemma}
    \uses{rChord}
    $\mathrm{rChord}$ maps an edge to itself if and only if the input vertex is a intersection 
    of {\it Inner} and {\it Outer}. 
\end{lemma}
\begin{proof}
    TODO
\end{proof}

Using these, we define the function $\mathrm{next}$

\begin{definition}\label{next}\lean{next}\leanok
    $\mathrm{next}: \mathrm{Dom} \to \mathrm{Dom}$ sends a pair of vertex and edge to the next one on the polygon.
    \[
        \mathrm{next}(V, E) = \mathrm{rChord}(\mathrm{rPoint}(V, E))
    \]
\end{definition}
It is obvious that
\begin{lemma}
    \lean{next_bijOn}
    \label{next_bijOn}
    \uses{next}
    \leanok
    $\mathrm{next}$ is a bijection from $\mathrm{Dom}$ to itself.
\end{lemma}
\begin{proof}
    $\mathrm{next}$ is a composition of two bijections
\end{proof}

With the definition ready, we now formalize the statement of a version of Poncelet's closure theorem
\begin{theorem}
    Given a two circles configuration in field $K$, if for a natural number $n$, for some 
    $(V, E) \in \mathrm{Dom}$ it holds that $(\mathrm{next})^n(V, E) = (V, E)$, then the same holds for 
    all $(V, E) \in \mathrm{Dom}$.
\end{theorem}

\chapter{Elliptic curve}

To prove the Poncelet's closure theorem for the two circles configuration, we will construct a bijection
between $\mathrm{Dom}$ and a certain elliptic curve $E$, and show that the function $mathrm{next}$ maps
to an addition in the group of the elliptic curve. We define the curve $E$ as follows

\begin{definition}\label{elliptic}\lean{elliptic}\leanok
    $E$ is an affine curve over $K$ defined by
    \[
    y^2 = x\left(r^2x^2 + \left(1-u^2-r^2\right)x + u^2\right)
    \]
\end{definition}

The curve $E$ is not strictly always an elliptic curve because it can be singular. Precisely speaking,
under the constraints $u \ne 0$ and $r \ne 0$, this curve is singular if and only of 
$(u + r) ^ 2 = 1$ or $(u - r) ^ 2 = 1$, meanin that the two circles are tangent to each other. In
general, the group law still holds for non-singular points on a singular cubic curve, so we don't 
treat this case separately for most parts. However, this will exclude the singular point from the 
bijection that we will construct, which we will discuss later.


\chapter{Two conics}